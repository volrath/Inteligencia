\documentclass[12pt, spanish]{article}
\usepackage[spanish, activeacute]{babel}
\usepackage[latin1]{inputenc}
\usepackage{graphicx}

\begin{document}
\title{Universidad Sim�n Bol�var \\ Inteligencia II \\ Tarea 1}
\author{
  Daniel Barreto - \#04-36723 \texttt{<daniel.barreto.n@gmail.com>} \\
  Kristoffer Pantic - \#05-38675 \texttt{<kristoffer.pantic@gmail.com>}
}
\maketitle

\section{Resumen}
\label{sec:resumen}
blabla

\section{Detalles de Implementaci�n}
\label{sec:di}
Para la realizaci�n de los puntos de la tarea se crearon tres tipos
abstractos de datos y algunas funciones auxiliares:

\subsection{Tipos abstractos de datos}

\textbf{Perceptron}\\
Esta clase representa una red neural sin capaz escondidas y con
�nicamente una neurona de salida, que mantiene un vector de pesos para
cada input que recibe. En esta clase se encuentran los m�todos
necesarios para evaluar el resultado de un vector de inputs y para
entrenarse con un conjunto de prueba.

La evaluaci�n devuelve un n�mero real que est� dado como el producto
punto del vector de inputs y el vector de pesos

\textbf{BooleanPerceptron}\\
Esta clase hereda los m�todos de la clase \textbf{Perceptron}, pero
sobreescribe la forma en la que se realiza la evaluaci�n de los inputs
para representar la evaluaci�n de una funcion \emph{threshold} y
devolver �nicamente 0 cuando la evaluaci�n real d� un n�mero negativo,
o 1 en el caso contrario.

\textbf{NeuralNetwork}\\
Bla...

\end{document}