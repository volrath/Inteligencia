\documentclass[10pt, spanish]{article}
\usepackage[spanish, activeacute]{babel}
\usepackage[latin1]{inputenc}
\usepackage{graphicx}

\begin{document}
\title{Universidad Sim�n Bol�var \\ Inteligencia II \\ Tarea 1}
\author{
  Daniel Barreto - \#04-36723 \texttt{<daniel.barreto.n@gmail.com>} \\
  Kristoffer Pantic - \#05-38675 \texttt{<kristoffer.pantic@gmail.com>}
}
\maketitle

\section{Resumen}
\label{sec:resumen}
blabla

\section{Detalles de Implementaci�n}
\label{sec:di}
Para la realizaci�n de los puntos de la tarea se crearon cuatro tipos
abstractos de datos y algunas funciones auxiliares:

\subsection{Tipos abstractos de datos}

\textbf{Perceptron}\\
Esta clase representa una red neural sin capaz escondidas y con
�nicamente una neurona de salida, que mantiene un vector de pesos para
cada input que recibe. En esta clase se encuentran los m�todos
necesarios para evaluar el resultado de un vector de inputs y para
entrenarse con un conjunto de prueba.

La evaluaci�n devuelve un n�mero real que est� dado como el producto
punto del vector de inputs y el vector de pesos\\

\textbf{BooleanPerceptron}\\
Esta clase hereda los m�todos de la clase \emph{Perceptron}, pero
sobreescribe la forma en la que se realiza la evaluaci�n de los inputs
para representar la evaluaci�n de una funcion \emph{threshold} y
devolver �nicamente 0 cuando la evaluaci�n real d� un n�mero negativo,
o 1 en el caso contrario.\\

\textbf{SigmoidNeuron}\\
Hereda igualmente de la clase \emph{Perceptron} para obtener su
atributo b�sico de la lista de pesos asociados a una lista de
inputs. La diferencia entre \emph{SigmoidNeuron} y \emph{Perceptron}
se encuentra en la evaluaci�n de un input dado, para el cual
\emph{SigmoidNeuron} retorna la funci�n sigmoidal aplicada sobre el
producto punto del vector de pesos y el vector de inputs.\\

\textbf{NeuralNetwork}\\
Utiliza SigmoidNeuron para generar las instancias de todas las
neuronas que conforman la red. Posee 3 atributos: \emph{n�mero de
  inputs}, \emph{lista de neuronas de la capa intermedia}, \emph{lista
  de neuronas de la capa de salida}.  Con dichos atributos se define
el entrenamiento con backpropagation como un m�todo de la clase d�nde
se utilizan listas de listas de pesos en vez de una matriz de pesos
para representar las fuerzas de las
conexiones entre cada neurona.\\

\subsection{Funciones Auxiliares}
\textbf{\texttt{training(neural\_network, training\_set,
    learning\_rate=.1,\\ max\_iterations=1000, reduce\_rate=False)}}\\
Es la funci�n principal del programa al momento de entrenar una red
neural, se encarga de realizar un m�ximo de iteraciones
(\texttt{max\_iterations}) sobre un conjunto de entrenamiento
(\texttt{training\_set}) dado, d�nde en cada iteraci�n se le pide a la
red neural (\texttt{neural\_network}) que entrene con
\emph{backpropagation} utilizando una taza de aprendizaje
(\texttt{learning\_rate}) y posiblemente una reducci�n de la misma en
cada iteraci�n (\texttt{reduce\_rate}).\\

\textbf{\texttt{test(neural\_network, test\_set)}}\\
Prueba el rendimiento de una red neural (\texttt{neural\_network})
sobre un conjunto de pruebas (\texttt{test\_set}). Devuelve un par de
listas de puntos; en la primera lista de puntos se encuentran todos
aquellos puntos que forman parte del \emph{�rea A}, y en la segunda
los que forman parte del\emph{�rea B}.\\

A parte de estas funciones, existen otras menos importantes como:\\
\texttt{get\_random\_set(set\_size)},
\texttt{load\_training\_set(file\_name)}, \texttt{plot(error\_log,
  test\_log)} que se utilizan para generar conjuntos random de pruebas,
cargar conjuntos de pruebas desde archivos y graficar la salida del
desempe�o de una red neural, respectivamente.

\section{An�lisis de Resultados}
\label{sec:analisis}

los resultados...

\end{document}